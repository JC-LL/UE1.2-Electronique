\chapter{Un monde {\it Cyber-Physique}}
\minitoc

\section{Rappels historiques}
\lettrine{I}l serait tentant d'affirmer que le monde est numérique. Mais ce serait là une simple
figure de style : une onde reste une onde et son champ électrique une grandeur belle et bien analogique, de même qu'une vitesse ou
un niveau de radioactivité. Mais force est de constater que toute notre vie est désormais sous l'emprise
de technologies effectivement numériques : téléphones portables, ordinateurs, objets connectés, pacemaker, etc. Dans ces dispositifs,
l'essentiel des traitements sont {\it numériques}. Il ne s'agit pas là d'une mince affaire --notamment pour vous, ex-taupins--, car
les mathématiques qui permettent de raisonner sur le traitement d'un signal numérique diffèrent des mathématiques traditionnelles. Vous avez découvert ces notions au cours de la première partie de l'UV. Par ailleurs, il existe un écart important entre les mathématiques relatives à ces traitements et celles relative à leur réalisation matérielle : toute une nouvelle
algèbre est nécessaire. Il s'agit de l'Algèbre de Boole.

%======================================================

\subsection{Babbage, Boole, Turing, Shannon et les autres}
\subsection{De Shockley au 4004}
\subsection{L'essor de la Microélectronique}

%======================================================
\section{Les merveilles du Numérique}

\subsection{Circuits correcteurs d'erreurs}

\subsection{Compression des données}

\subsection{Traitement Radar}

%====================================================
\section{Les nouveaux challenges du Numérique}
\subsection{Maîtriser la complexité et le risque industriel}
\subsection{La {\it cyber-sécurité}}

\section{Conclusion}
