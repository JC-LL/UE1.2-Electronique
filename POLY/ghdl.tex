\chapter*{Utilisation de GHDL}


\section{Introduction}
GHDL est un simulateur open source développé par Tristan Gingold, un ingénieur français.
A la différence d'outils professionnels comme Modelsim de la société Mentor Graphics, GHDL s'utilise
uniquement en ligne de commande. GHDL peut être être utilisé en synergie avec un second outil appelé GTKWave dans le but de visualiser des chronogrammes.
Nous rappelons ici les commandes utiles qui permettent de simuler un circuit simple.
D'autres commandes sont disponibles mais ne seront pas traitées ici.

\section{Commandes essentielles}


GHDL s'utilise en ligne de commande comme n'importe quel compilateur.
Le 'G' de GHDL fait référence au projet GNU lié à l'Open source dont le fer de lance est GCC, le compilateur pour le langage C (et bien d'autres langages !).
D'ailleurs GHDL utilise le même backend que GCC : il génère les mêmes fichiers binaires "*.o". A la date de rédaction (2017), une version de GHDL génère un bitcode
LLVM, alternative à GCC.

Les options les plus utiles sont les suivantes, utilisée dans cet ordre :
\begin{itemize}
  \item \textbf{ghdl \-a nom\_fichier.vhd} : Analyse le fichier et génère un binaire (faites ls -a nom\_fichier.o pour le voir si vous le souhaitez). Généralement il faut compiler plusieurs fichiers avant de passer à l'étape suivante.
 \item \textbf{ghdl \-e nom\_de\_l'entité\_simulable}  : Elaboration (ou Edition de lien), qui crée le simulateur compilé : c'est un exécutable comme un autre. Il embarque un moteur de simulation à événement discret.
\item \textbf{ghdl \-r nom\_de\_l'entité\_simulable}  : Run de l'exécutable précédent. Il est tout à fait possible de lancer l'exécutable sans passer par cette commande.
\end{itemize}

Généralement, on cherche à enregistrer les formes d'ondes (waveforms ou chronogrammes) dans un fichier, pour une visualisation post-mortem (c'est un désavantage de ghdl par rapport à Modelsim, qui permet d'arrêter une simulation et de la reprendre etc). La manière de faire consiste à modifier la dernière commande précédente :

    \textbf{ghdl -r <nom\_de\_l'entité\_simulable> --wave=<file.ghw>}

Le fichier au format ghw est ensuite lisible par le logiciel Gtkwave.

    \textbf{gtkwave file.ghw}

Le logiciel Gtkwave permet de naviguer dans la hiérarchie de votre design, et de sélectionner les signaux que vous souhaitez visualiser.
Cette sélection peut être enregistrée pour une visualisation future, en sauvant un fichier .sav (passer par le menu de gtkwave). On pourra alors
relancer la simulation en demandant à GTKWave d'afficher automatiquement ces signaux.

   \textbf{gtkwave file.ghw signaux.sav}

Au final, il est possible d'automatiser le processus de compilation-simulation-visualisation en copiant les commandes précédentes dans un script.
N'oubliez pas de rendre ce script exécutable, en lui donnant les bons droits (chmod +x nom\_script).
