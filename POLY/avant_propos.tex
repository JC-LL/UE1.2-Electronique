\chapter*{Avant propos}

Ce fascicule propose une introduction à l'Electronique Numérique aux étudiants de première année
de l'ENSTA Bretagne. Dans un temps très contraint (moins de 10 séances), nous abordons des notions
aussi fondamentales que la représentation des données numériques (les nombres, mais pas seulement...),
la logique Booléenne, les portes logiques élémentaires, les circuits séquentiels synchrones, etc. Nous avons conservé
un premier chapitre consacré à des rappels de physique, qui aboutissent à l'idée du transistor en commutation : il est là
moins pour donner des bases de Micro-électronique, que pour simplement {\it évoquer} l'idée d'un continuum dans l'Histoire de l'Electronique
et faire appel à quelques réminiscences  éventuelles en matière de cristallographie.Il permet également d'insister sur la {\it montée en abstraction} dans le domaine de l'Electronique.\\

A l'inverse, nous apportons un soin particulier lors de la découverte des machines d'états finis (FSM) dont on connaît l'importance dans le domaine
des systèmes embarqués, y compris lorsque ceux-ci sont à "logiciels prépondérants" : nous sommes  d'emblée
dans la capacité de modéliser des systèmes numériques et d'en dériver mécaniquement les équations constitutives.
Au passage, nous prenons le temps de définir la notion de causalité dans de telles machines : il s'agit là
d'une première intrusion de techniques formelles rigoureuses, dans un univers informatique qui en parait -- pour l'étudiant pythoniste débutant-- le plus souvent
dépourvu.\\

Nous prenons enfin le parti de profiter de ce cours pour introduire le langage VHDL, support incontournable
de la conception de circuits numériques : simulation et synthèse sur FPGA sont abordées. Nous avons dû nous restreindre
à un sous ensemble modeste du langage, correspondant au niveau logique : équations et bascules. Le niveau RTL est seulement abordé ici.
Le simulateur GHDL retenu est open-source : il s'utilise naturellement en ligne de commande et permet selon nous de banaliser de telles simulations.
Enfin, lors d'une ultime séance de travaux pratiques, nous synthétisons un petit circuit sur FPGA, à l'aide de scripts TCL.
Là encore, notre souhait est à la fois de faire acquérir aux élèves des notions de base, mais également de lever un coin du voile
sur le domaine des systèmes reconfigurables : ce domaine fascinant est en pleine croissance. Il accompagne désormais l'essor du Cloud Computing et tous les domaines
métier, plus traditionnels, de l'Embarqué : automobile, multimédia, militaire etc.\\

Toutes ces notions seront bien entendu approfondies en deuxième année.\\

{\it Merci à tous les étudiants qui pourront apporter des retours constructifs à l'amélioration continue de ce fascicule.}
